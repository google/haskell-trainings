%%%%%%%%%%%%%%%%%%%%%%%%%%%%%%%%%%%%%%%%%%%%%%%%%%%%%%%%%%%%%%%%%%%%%%%%%%%%
%% Header

\documentclass[17pt]{beamer}

\usepackage[english]{babel}
\usepackage{amssymb}
\usepackage{beamerthemesplit}
\usepackage{calc}
\usepackage{colortbl}
\usepackage{floatflt}
\usepackage{graphicx}
\usepackage{listings}
\usepackage{soul}
\usepackage{xunicode}

\usepackage[no-math]{fontspec}
\usepackage{fdsymbol}

\title{Haskell 102}
\author{nicuveo@}
%\institute{Google}
\date{\small\today}



%%%%%%%%%%%%%%%%%%%%%%%%%%%%%%%%%%%%%%%%%%%%%%%%%%%%%%%%%%%%%%%%%%%%%%%%%%%%
%% Customization

%%
%% theme.tex
%% Made by nicuveo <antoine.jp.leblanc@gmail.com>
%%


%%%%%%%%%%%%%%%%%%%%%%%%%%%%%%%%%%%%%%%%%%%%%%%%%%%%%%%%%%%%%%%%%%%%%%%%%%%%
%% Base theme

\mode<presentation>{}
\usecolortheme{whale}
%\usepackage{fontspec}
\usepackage{etoolbox}
\setsansfont[Ligatures=TeX, % recommended
             UprightFont={*-Regular},
             ItalicFont={*-Regular},
             BoldFont={*-Bold},
             BoldItalicFont={*-Bold}]
             {Yanone Kaffeesatz}



%%%%%%%%%%%%%%%%%%%%%%%%%%%%%%%%%%%%%%%%%%%%%%%%%%%%%%%%%%%%%%%%%%%%%%%%%%%%
%% Colors

% define

\definecolor{google-r}{RGB}{234,  67,  53}
\definecolor{google-g}{RGB}{ 52, 168,  83}
\definecolor{google-b}{RGB}{ 66, 133, 244}
\definecolor{google-y}{RGB}{251, 188,   5}

\definecolor{my-color-1}{RGB}{ 66, 133, 244} % primary
\definecolor{my-color-2}{RGB}{234,  67,  53} % lighter
\definecolor{my-color-3}{RGB}{  0,   0,   0} % unknown
\definecolor{my-color-4}{RGB}{  0,   0,   0} % darker

\definecolor{tab-1}{rgb}{0.04,0.34,0.58}
\definecolor{tab-2}{rgb}{0.36,0.56,0.72}
\definecolor{tab-3}{rgb}{0.68,0.78,0.86}


% set

\setbeamercolor{structure}{fg=my-color-1}
\setbeamercolor{alerted text}{fg=my-color-2}

\setbeamercolor{palette primary}   {fg=white, bg=my-color-1}
\setbeamercolor{palette secondary} {fg=black, bg=my-color-2}
\setbeamercolor{palette tertiary}  {fg=white, bg=my-color-3}
\setbeamercolor{palette quaternary}{fg=white, bg=my-color-4}

\setbeamercolor{page number in head/foot} {fg=black, bg=white}
\setbeamercolor{icon in head/foot} {fg=black, bg=white}

\setbeamercolor*{separation line}{}
\setbeamercolor*{fine separation line}{}



%%%%%%%%%%%%%%%%%%%%%%%%%%%%%%%%%%%%%%%%%%%%%%%%%%%%%%%%%%%%%%%%%%%%%%%%%%%%
%% Templates

\newtoggle{showpagenumber}

\setbeamertemplate{headline}[default]
\defbeamertemplate{footline}{mine}[2]
{
  \leavevmode%
  \hbox
  {%
    \hskip .020\paperwidth

    \begin{beamercolorbox}[wd=.160\paperwidth,ht=3ex,dp=3ex,center]{icon in head/foot}
      \pgfimage[mask=#2,interpolate=true,height=14pt]{#1}
    \end{beamercolorbox}%

    \hskip .720\paperwidth

    \begin{beamercolorbox}[wd=.100\paperwidth,ht=3ex,dp=3ex,center]{page number in head/foot}
      \usebeamerfont{page number in head/foot}
      \iftoggle{showpagenumber}{
        \insertframenumber{} / \inserttotalframenumber{}
      }{}
      \end{beamercolorbox}%
  }%
  \vskip0pt%
}


\newcounter{SectionColorCounter}
\AtBeginSection[]
{
  \ifnum\value{SectionColorCounter}=0
    \setbeamercolor{palette primary}{fg=white, bg=google-g}
    \setbeamercolor{frametitle}{fg=white, bg=google-g}
    \setbeamercolor{structure}{fg=google-g}
    \setbeamercolor{alerted text}{fg=google-g}
  \fi
  \ifnum\value{SectionColorCounter}=1
    \setbeamercolor{palette primary}{fg=white, bg=google-b}
    \setbeamercolor{frametitle}{fg=white, bg=google-b}
    \setbeamercolor{structure}{fg=google-b}
    \setbeamercolor{alerted text}{fg=google-b}
  \fi
  \ifnum\value{SectionColorCounter}=2
    \setbeamercolor{palette primary}{fg=white, bg=google-r}
    \setbeamercolor{frametitle}{fg=white, bg=google-r}
    \setbeamercolor{structure}{fg=google-r}
    \setbeamercolor{alerted text}{fg=google-r}
  \fi
  \ifnum\value{SectionColorCounter}=3
    \setbeamercolor{palette primary}{fg=white, bg=google-y}
    \setbeamercolor{frametitle}{fg=white, bg=google-y}
    \setbeamercolor{structure}{fg=google-y}
    \setbeamercolor{alerted text}{fg=google-y}
  \fi

  \stepcounter{SectionColorCounter}
  \ifnum\value{SectionColorCounter}=4
    \setcounter{SectionColorCounter}{0}
  \fi
}

% \AtBeginSection[]
% {
%    \begin{frame}
%        \frametitle{Outline}
%        \tableofcontents[currentsection,hideothersubsections]
%    \end{frame}
% }

\newcommand{\setfooterlogo}[2]
{
  \setbeamertemplate{footline}[mine]{#1}{#2}
}

%\pgfdeclaremask{masklogo}{img/logo}
\newcommand\defaultlogo{\setfooterlogo{img/google}{}}
\defaultlogo

\setbeamersize{text margin left=10pt,text margin right=10pt}



%%%%%%%%%%%%%%%%%%%%%%%%%%%%%%%%%%%%%%%%%%%%%%%%%%%%%%%%%%%%%%%%%%%%%%%%%%%%
%% Code environnement

\definecolor{hsk-comment} {gray}{0.5}
\colorlet{hsk-built-ins}{google-g}
\colorlet{hsk-types}    {google-b}
\colorlet{hsk-operators}{google-g}
\colorlet{hsk-keywords} {google-r}
\colorlet{hsk-consts}   {google-g}
\colorlet{hsk-strings}  {google-y!80!black}

\lstdefinelanguage{ColorHaskell} {
        basicstyle=\ttfamily\footnotesize,
        sensitive=true,
        morecomment=[l][\color{hsk-comment}\ttfamily\footnotesize]{--},
        morecomment=[s][\color{hsk-comment}\ttfamily\footnotesize]{\{-}{-\}},
        morestring=[b]",
        stringstyle=\color{hsk-strings},
        showstringspaces=false,
        numberstyle=none,
        showspaces=false,
        breaklines=true,
        showtabs=false,
        emph=
        {[1]
                abs,acos,acosh,all,and,any,appendFile,approxRational,asTypeOf,asin,
                asinh,atan,atan2,atanh,basicIORun,break,catch,ceiling,chr,compare,concat,concatMap,
                const,cos,cosh,curry,cycle,decodeFloat,denominator,digitToInt,divMod,drop,
                dropWhile,either,encodeFloat,enumFrom,enumFromThen,enumFromThenTo,enumFromTo,
                error,even,exp,exponent,fail,filter,flip,floatDigits,floatRadix,floatRange,floor,
                fmap,foldl,foldl1,foldr,foldr1,fromDouble,fromEnum,fromInt,fromInteger,fromIntegral,
                fromRational,fst,gcd,getChar,getContents,getLine,head,id,inRange,index,init,intToDigit,
                interact,ioError,isAlpha,isAlphaNum,isAscii,isControl,isDenormalized,isDigit,isHexDigit,
                isIEEE,isInfinite,isLower,isNaN,isNegativeZero,isOctDigit,isPrint,isSpace,isUpper,iterate,
                last,lcm,length,lex,lexDigits,lexLitChar,lines,log,logBase,lookup,map,mapM,mapM_,max,
                maxBound,maximum,maybe,min,minBound,minimum,negate,not,null,numerator,odd,
                or,ord,otherwise,pi,pred,primExitWith,print,product,properFraction,putChar,putStr,putStrLn,
                quotRem,range,rangeSize,read,readDec,readFile,readFloat,readHex,readIO,readInt,readList,readLitChar,
                readLn,readOct,readParen,readSigned,reads,readsPrec,realToFrac,recip,repeat,replicate,return,
                reverse,round,scaleFloat,scanl,scanl1,scanr,scanr1,sequence,sequence_,show,showChar,showInt,
                showList,showLitChar,showParen,showSigned,showString,shows,showsPrec,significand,signum,sin,
                sinh,snd,span,splitAt,sqrt,subtract,succ,sum,tail,take,takeWhile,tan,tanh,threadToIOResult,toEnum,
                toInt,toInteger,toLower,toRational,toUpper,truncate,uncurry,undefined,unlines,until,unwords,unzip,
                unzip3,userError,words,writeFile,zip,zip3,zipWith,zipWith3,listArray,doParse
        },
        emphstyle={[1]\color{hsk-built-ins}},
        emph=
        {[2]
                FilePath,IOError,Bool,Char,Double,Either,Float,IO,Integer,Int,Maybe,Ordering,Rational,Ratio,ReadS,ShowS,String, Word8,InPacket
        },
        emphstyle={[2]\color{hsk-types}},
        emph=
        {[3]
                case,class,data,deriving,do,else,if,import,in,infixl,infixr,instance,let,
                module,of,primitive,then,type,where
        },
        emphstyle={[3]\color{hsk-keywords}\textbf},
        emph=
        {[4]
                quot,rem,div,mod,elem,notElem,seq
        },
        emphstyle={[4]\color{hsk-operators}\textbf},
        emph=
        {[5]
                EQ,False,GT,Just,LT,Left,Nothing,Right,True,Show,Eq,Ord,Num,Enum,Bounded
        },
        emphstyle={[5]\color{hsk-consts}\textbf}
}

\lstnewenvironment{code}
    {\lstset{language=ColorHaskell,basicstyle=\footnotesize\ttfamily}%
      \csname lst@SetFirstLabel\endcsname}
    {\csname lst@SaveFirstLabel\endcsname}
    \lstset{
      basicstyle=\footnotesize\ttfamily,
      flexiblecolumns=false,
      basewidth={0.5em,0.45em},
      literate={\\}{{$\lambda$}}1
               {\\\\}{{\char`\\\char`\\}}1
               {->}{{$\rightarrow$}}2 {<-}{{$\leftarrow$}}2
               {=>}{{$\Rightarrow$}}2
               {>>}{{>>}}2 {>>=}{{>>=}}3 {>=>}{{>=>}}3
               {|}{{$\mid$}}1
    }

\def\inlinecode{\lstinline[language=ColorHaskell,
      basicstyle=\footnotesize\ttfamily,
      flexiblecolumns=false,
      basewidth={0.5em,0.45em},
      literate={\\}{{$\lambda$}}1
               {\\\\}{{\char`\\\char`\\}}1
               {->}{{$\rightarrow$}}2 {<-}{{$\leftarrow$}}2
               {=>}{{$\Rightarrow$}}2
               {>>}{{>>}}2 {>>=}{{>>=}}3 {>=>}{{>=>}}3
               {|}{{$\mid$}}1]}

\newcommand<>{\ic}{\inlinecode}



%%%%%%%%%%%%%%%%%%%%%%%%%%%%%%%%%%%%%%%%%%%%%%%%%%%%%%%%%%%%%%%%%%%%%%%%%%%%
%% Fonts

\setbeamerfont{frametitle}{size=\small}
\setbeamerfont{structure}{series=\bfseries}



%%%%%%%%%%%%%%%%%%%%%%%%%%%%%%%%%%%%%%%%%%%%%%%%%%%%%%%%%%%%%%%%%%%%%%%%%%%%
%% Settings

\setbeamertemplate{navigation symbols}{}

\bibliographystyle{apalike}

\mode<all>


\renewcommand{\(}[1]{\begin{columns}[#1]}
\renewcommand{\)}{\end{columns}}
\newcommand{\<}[1]{\begin{column}{#1}}
\renewcommand{\>}{\end{column}}

\newcommand{\Split}[3][.5]{%
\({c}%
  \<{{#1}\linewidth}%
  \begin{minipage}[c][.8\textheight]{\linewidth}%
  \begin{center}%
    {#2}%
  \end{center}%
  \end{minipage}%
  \>%
  \<{\linewidth-{#1}\linewidth}%
  \begin{minipage}[c][.8\textheight]{\linewidth}%
  \begin{center}%
    {#3}%
  \end{center}%
  \end{minipage}
  \>%
\)%
}

\newcommand<>{\Image}[3][]{%
\IfFileExists{#3.png}{%
\includegraphics#4[width=\linewidth,height=#2,keepaspectratio,#1]{#3}%
}{%
\IfFileExists{#3.jpg}{%
\includegraphics#4[width=\linewidth,height=#2,keepaspectratio,#1]{#3}%
}{%
\includegraphics#4[width=\linewidth,height=#2,keepaspectratio,#1]{img/placeholder}%
}}}

\newenvironment{TopAlign}[2][1]{\begin{minipage}[t][#2\textheight]{#1\linewidth}}{\end{minipage}}

\newcommand{\pc}[1]{{\footnotesize\texttt{#1}}}
\newcommand<>{\opc}[1]{\only#2{\pc{#1}}}
\newcommand<>{\upc}[1]{\uncover#2{\pc{#1}}}
\newcommand<>{\oic}[1]{\only#2{\ic{#1}}}
\newcommand<>{\uic}[1]{\uncover#2{\ic{#1}}}

\def\shruggie{\texttt{\raisebox{0.75em}{\char`\_}\char`\\\char`\_\kern-0.5ex(\kern-0.25ex\raisebox{0.25ex}{\rotatebox{45}{\raisebox{-.75ex}"\kern-1.5ex\rotatebox{-90})}}\kern-0.5ex)\kern-0.5ex\char`\_/\raisebox{0.75em}{\char`\_}}}


%%%%%%%%%%%%%%%%%%%%%%%%%%%%%%%%%%%%%%%%%%%%%%%%%%%%%%%%%%%%%%%%%%%%%%%%%%%%
%% Document

\begin{document}



%% Title frame

\togglefalse{showpagenumber}
\begin{frame}[fragile]
  \titlepage
\end{frame}
\toggletrue{showpagenumber}
\setcounter{framenumber}{0}



%% Intro

\section{Intro}

\begin{frame}
\frametitle{Goals}
\Split{%
  \begin{itemize}
  \item 101: basic skills
    \begin{itemize}
    \item Reading function types
    \item Pattern matching
    \item Data structures
  \end{itemize}
  \end{itemize}
  \begin{itemize}
  \item 102: first project
    \begin{itemize}
    \item Genericity
    \item IO and do notation
    \item Build a game!
  \end{itemize}
  \end{itemize}
}{%
  \Image{3cm}{img/success}
}%
\end{frame}

\begin{frame}
\frametitle{Roadmap}
\Split{%
\begin{TopAlign}{.52}
  \begin{itemize}
  \item 101 Recap
  \item Remaining obstacles
  \item Typeclasses overview
  \item \alt<2>{Typeclasses}{Common examples}
  \item \alt<2>{Typeclasses}{Advanced syntax}
  \item<2> Typeclasses
  \item<2> Typeclasses\ldots
  \end{itemize}
\end{TopAlign}
}{%
  \alt<2>{%
    \Image{4cm}{img/pinkie-ko}
  }{%
    \Image{4cm}{img/pinkie-ok}
  }
}%
\end{frame}

\begin{frame}
\frametitle{Checklist}
\Split[.6]{%
  \begin{itemize}
  \item Haskell 101
  \item download the codelab
  \item apt-get install haskell-platform
  \end{itemize}
}{%
  \Image{4cm}{img/checklist}
}%
\end{frame}



%% Recap 101

\section{Recap}

\begin{frame}
\frametitle{Curried functions, partial application}
\begin{center}
\begin{tabular}{ l c r }
  \uncover<2->{\\\inlinecode{f}      &\inlinecode{::}&\inlinecode{Int -> ( Int -> [Int] )}}
  \uncover<1->{\\\inlinecode{f}      &\inlinecode{::}&\inlinecode{Int ->   Int -> [Int]}}
  \uncover<3->{\\\inlinecode{f 1}    &\inlinecode{::}&\inlinecode{         Int -> [Int]}}
  \uncover<5->{\\\inlinecode{f 1 2}  &\inlinecode{::}&\inlinecode{                [Int]}}
  \uncover<4->{\\\inlinecode{(f 1) 2}&\inlinecode{::}&\inlinecode{                [Int]}}
\end{tabular}
~\\~\\~\\
\end{center}
\end{frame}

\begin{frame}[fragile]
\frametitle{Type constructors}
\begin{center}
\begin{uncoverenv}<1->
  \begin{code}
      -- enum
      data Bool  = False | True
      data Color = Red | Green | Blue
  \end{code}
\end{uncoverenv}
\begin{uncoverenv}<2->
  \begin{code}
      -- struct
      data Point = Point { x :: Double, y :: Double }
  \end{code}
\end{uncoverenv}
\begin{uncoverenv}<3->
  \begin{onlyenv}<-3>
    \begin{code}
      -- a bit more interesting
      data Minutes = Minutes Int
      data Maybe a = Nothing | Just a
      data List a  = Nil | Cell a (List a)
    \end{code}
  \end{onlyenv}
  \begin{onlyenv}<4>
    \begin{code}
      -- a bit more interesting
      data Minutes = Minutes Int
      data Maybe a = Nothing | Just a
      data [a]     = [] | (a:[a])
    \end{code}
  \end{onlyenv}
\end{uncoverenv}
\end{center}
\end{frame}

\begin{frame}[fragile]
\frametitle{"Deconstructors" and pattern matching}
\begin{center}
\begin{uncoverenv}<1->
  \begin{code}
      not :: Bool -> Bool
      not True  = False
      not False = True
  \end{code}
\end{uncoverenv}
\begin{uncoverenv}<2->
  \begin{code}
      magnitude :: Point -> Double
      magnitude (Point x y) = sqrt $ x^2 + y^2
  \end{code}%$
\end{uncoverenv}
\begin{uncoverenv}<3->
  \begin{code}
      length :: [a] -> Int
      length []     = 0
      length (_:xs) = 1 + length xs
  \end{code}
\end{uncoverenv}
\end{center}
\end{frame}

\begin{frame}
\frametitle{We know}
\begin{center}
\begin{minipage}[c]{.8\linewidth}
\begin{itemize}
  \item how to read function types
  \item how to declare new types
  \item how to do pattern matching
\end{itemize}
\end{minipage}
\end{center}
\end{frame}



%% Shortcomings

\section{Obstacles}

\begin{frame}[fragile]
\frametitle{Our types are limited}
\Split[.64]{%
\begin{TopAlign}{.5}
\begin{flushleft}
               \footnotesize\texttt{\$ ghci}                                                ~\\
\uncover<1->  {\footnotesize\texttt{> \uncover<2-> {\ic{data Color = Red | Green | Blue}}}} ~\\
\uncover<3->  {\footnotesize\texttt{> \uncover<4-> {\ic{let mycolor = Red}}}}               ~\\
\begin{onlyenv}<5-7>
\uncover<5-7> {\footnotesize\texttt{> \uncover<6-7>{\ic{mycolor}}}}                         ~\\
\uncover<7>   {\footnotesize\texttt{~~~~No instance for (Show Color)}}                      ~\\
\uncover<7>   {\footnotesize\texttt{~~~~arising from a use of `print'}}                     ~\\
\end{onlyenv}
\begin{onlyenv}<8-10>
\uncover<8-10>{\footnotesize\texttt{> \uncover<9-10>{\ic{mycolor == Red}}}}                 ~\\
\uncover<10>  {\footnotesize\texttt{~~~~No instance for (Eq Color)}}                        ~\\
\uncover<10>  {\footnotesize\texttt{~~~~arising from a use of `(==)'}}                      ~\\
\end{onlyenv}
\end{flushleft}
\end{TopAlign}
}{%
\Image<1-6,8-9>{4cm}{img/rarity-surprised}
\Image<7>{4.2cm}{img/rarity-angry}
\Image<10>{4cm}{img/rarity-confused}
}
\end{frame}

\begin{frame}[fragile]
\frametitle{Type parameters constraints?}
~\\
\begin{code}
  sumI :: [Int] -> Int        sumD :: [Double] -> Double
  sumI = foldl' (+) 0         sumD = foldl' (+) 0
\end{code}
\pause~\\
\begin{code}
                  sum :: [a] -> a
                  sum = foldl' (+) 0
\end{code}
\pause
\vspace{-1.5cm}
\Split[0.55]{%
\begin{flushleft}
\footnotesize
\texttt{~~No instance for (Num a)}\\
\texttt{~~arising from a use of `(+)'}
\end{flushleft}
}{%
\vspace{1cm}
\Image{4cm}{img/derpy-shrug}
}
\end{frame}

\begin{frame}[fragile]
  \frametitle{Cascading context}
  \begin{TopAlign}{1.0}
  \begin{onlyenv}<1>
  \begin{code}
getUser        :: Id   -> Maybe User
getNextOfKin   :: User -> Maybe Id
getPhoneNumber :: User -> Maybe PhoneNumber
  \end{code}
  \begin{center}
  \vspace{2.22cm}
  \hspace{4cm}\Image{4cm}{img/fluttershy-walking}
  \end{center}
  \end{onlyenv}
  \begin{onlyenv}<2>
  \begin{code}
getUser        :: Id   -> Maybe User
getNextOfKin   :: User -> Maybe Id
getPhoneNumber :: User -> Maybe PhoneNumber

getNextOfKinPhoneNumber :: Id -> Maybe PhoneNumber
  \end{code}
  \begin{center}
  \vspace{.57cm}
  \hspace{3cm}\Image{4cm}{img/fluttershy-walking}
  \end{center}
  \end{onlyenv}
  \begin{onlyenv}<3>
  \begin{code}
getUser        :: Id   -> Maybe User
getNextOfKin   :: User -> Maybe Id
getPhoneNumber :: User -> Maybe PhoneNumber

getNextOfKinPhoneNumber :: Id -> Maybe PhoneNumber
getNextOfKinPhoneNumber userId =
  \end{code}
  \begin{center}
  \vspace{-.2cm}
  \Image{4cm}{img/fluttershy-curious}
  \end{center}
  \end{onlyenv}
  \begin{onlyenv}<4>
  \begin{code}
getUser        :: Id   -> Maybe User
getNextOfKin   :: User -> Maybe Id
getPhoneNumber :: User -> Maybe PhoneNumber

getNextOfKinPhoneNumber :: Id -> Maybe PhoneNumber
getNextOfKinPhoneNumber userId = case getUser userId of
  \end{code}
  \begin{center}
  \vspace{-.2cm}
  \Image{4cm}{img/fluttershy-curious}
  \end{center}
  \end{onlyenv}
  \begin{onlyenv}<5>
  \begin{code}
getUser        :: Id   -> Maybe User
getNextOfKin   :: User -> Maybe Id
getPhoneNumber :: User -> Maybe PhoneNumber

getNextOfKinPhoneNumber :: Id -> Maybe PhoneNumber
getNextOfKinPhoneNumber userId = case getUser userId of
    Nothing   -> Nothing
  \end{code}
  \begin{center}
  \vspace{0.24cm}
  \Image{3cm}{img/fluttershy-scared}
  \end{center}
  \end{onlyenv}
  \begin{onlyenv}<6>
  \begin{code}
getUser        :: Id   -> Maybe User
getNextOfKin   :: User -> Maybe Id
getPhoneNumber :: User -> Maybe PhoneNumber

getNextOfKinPhoneNumber :: Id -> Maybe PhoneNumber
getNextOfKinPhoneNumber userId = case getUser userId of
    Nothing   -> Nothing
    Just user ->
  \end{code}
  \begin{center}
  \vspace{-.25cm}
  \Image{3cm}{img/fluttershy-scared}
  \end{center}
  \end{onlyenv}
  \begin{onlyenv}<7>
  \begin{code}
getUser        :: Id   -> Maybe User
getNextOfKin   :: User -> Maybe Id
getPhoneNumber :: User -> Maybe PhoneNumber

getNextOfKinPhoneNumber :: Id -> Maybe PhoneNumber
getNextOfKinPhoneNumber userId = case getUser userId of
    Nothing   -> Nothing
    Just user -> case getNextOfKin user of
  \end{code}
  \begin{center}
  \vspace{-.25cm}
  \Image{3cm}{img/fluttershy-scared}
  \end{center}
  \end{onlyenv}
  \begin{onlyenv}<8>
  \begin{code}
getUser        :: Id   -> Maybe User
getNextOfKin   :: User -> Maybe Id
getPhoneNumber :: User -> Maybe PhoneNumber

getNextOfKinPhoneNumber :: Id -> Maybe PhoneNumber
getNextOfKinPhoneNumber userId = case getUser userId of
    Nothing   -> Nothing
    Just user -> case getNextOfKin user of
        Nothing          -> Nothing
  \end{code}
  \begin{center}
  \vspace{.09cm}
  \Image{2.2cm}{img/fluttershy-flat}
  \end{center}
  \end{onlyenv}
  \begin{onlyenv}<9>
  \begin{code}
getUser        :: Id   -> Maybe User
getNextOfKin   :: User -> Maybe Id
getPhoneNumber :: User -> Maybe PhoneNumber

getNextOfKinPhoneNumber :: Id -> Maybe PhoneNumber
getNextOfKinPhoneNumber userId = case getUser userId of
    Nothing   -> Nothing
    Just user -> case getNextOfKin user of
        Nothing          -> Nothing
        Just nextOfKinId ->
  \end{code}
  \begin{center}
  \vspace{-.4cm}
  \Image{2.2cm}{img/fluttershy-flat}
  \end{center}
  \end{onlyenv}
  \begin{onlyenv}<10>
  \begin{code}
getUser        :: Id   -> Maybe User
getNextOfKin   :: User -> Maybe Id
getPhoneNumber :: User -> Maybe PhoneNumber

getNextOfKinPhoneNumber :: Id -> Maybe PhoneNumber
getNextOfKinPhoneNumber userId = case getUser userId of
    Nothing   -> Nothing
    Just user -> case getNextOfKin user of
        Nothing          -> Nothing
        Just nextOfKinId -> case getUser nextOfKinId of
  \end{code}
  \begin{center}
  \vspace{-.4cm}
  \Image{2.2cm}{img/fluttershy-flat}
  \end{center}
  \end{onlyenv}
  \begin{onlyenv}<11>
  \begin{code}
getUser        :: Id   -> Maybe User
getNextOfKin   :: User -> Maybe Id
getPhoneNumber :: User -> Maybe PhoneNumber

getNextOfKinPhoneNumber :: Id -> Maybe PhoneNumber
getNextOfKinPhoneNumber userId = case getUser userId of
    Nothing   -> Nothing
    Just user -> case getNextOfKin user of
        Nothing          -> Nothing
        Just nextOfKinId -> case getUser nextOfKinId of
            Nothing        -> Nothing
  \end{code}
  \begin{center}
  \vspace{-.15cm}
  \Image{4cm}{img/fluttershy-screaming}
  \end{center}
  \end{onlyenv}
  \begin{onlyenv}<12>
  \begin{code}
getUser        :: Id   -> Maybe User
getNextOfKin   :: User -> Maybe Id
getPhoneNumber :: User -> Maybe PhoneNumber

getNextOfKinPhoneNumber :: Id -> Maybe PhoneNumber
getNextOfKinPhoneNumber userId = case getUser userId of
    Nothing   -> Nothing
    Just user -> case getNextOfKin user of
        Nothing          -> Nothing
        Just nextOfKinId -> case getUser nextOfKinId of
            Nothing        -> Nothing
            Just nextOfKin ->
  \end{code}
  \begin{center}
  \vspace{-.64cm}
  \Image{4cm}{img/fluttershy-screaming}
  \end{center}
  \end{onlyenv}
  \begin{onlyenv}<13>
  \begin{code}
getUser        :: Id   -> Maybe User
getNextOfKin   :: User -> Maybe Id
getPhoneNumber :: User -> Maybe PhoneNumber

getNextOfKinPhoneNumber :: Id -> Maybe PhoneNumber
getNextOfKinPhoneNumber userId = case getUser userId of
    Nothing   -> Nothing
    Just user -> case getNextOfKin user of
        Nothing          -> Nothing
        Just nextOfKinId -> case getUser nextOfKinId of
            Nothing        -> Nothing
            Just nextOfKin -> getPhoneNumber nextOfKin
  \end{code}
  \begin{center}
  \vspace{-.64cm}
  \Image{4cm}{img/fluttershy-screaming}
  \end{center}
  \end{onlyenv}
  \end{TopAlign}
\end{frame}

\begin{frame}
\frametitle{IO}
\begin{TopAlign}{.1}
\begin{itemize}
  \item<1-> Can't apply regular functions on IO values
  \item<7-> Can't get values out of IO
  \item<9-> Can't pattern match on IO
\end{itemize}
\end{TopAlign}
\Split[0.6]{%
\begin{onlyenv}<2-6>
  \begin{flushleft}
               \footnotesize\texttt{\$ ghci}                                                  ~\\
  \uncover<2->{\footnotesize\texttt{> \uncover<3-> {\ic{let name = getLine -- IO String}}}}   ~\\
  \uncover<4->{\footnotesize\texttt{> \uncover<5-> {\ic{putStr \$ "Hello " ++ name ++ "!"}}}} ~\\
  \uncover<6->{\footnotesize\texttt{~~~~Expected type: `String'}}                             ~\\
  \uncover<6->{\footnotesize\texttt{~~~~~~Actual type: `IO String'}}                          ~\\
  \uncover<6->{\footnotesize\texttt{~~~~In second argument of (++)}}                          ~\\
  \end{flushleft}
\end{onlyenv}
\only<8>{IO is impure}
\only<10>{IO's implementation\\details are hidden}
}{%
\Image<1,7>{4.2cm}{img/twilight-thinking}
\Image<2-5>{4cm}{img/twilight-focus}
\Image<6>{4.2cm}{img/twilight-scared}
\only<8>{\vspace{0.2cm}}
\Image<8>{3.8cm}{img/twilight-disgust}
\only<9>{\vspace{0.2cm}}
\Image<9>{3.8cm}{img/twilight-jaded-2}
\only<10>{\vspace{0.2cm}\hspace{-0.5cm}}
\Image<10>{3.8cm}{img/twilight-jaded}
}
\end{frame}

\begin{frame}
\frametitle{We don't know}
\begin{center}
\begin{minipage}[c]{.8\linewidth}
\begin{itemize}
  \item how to extend our data types
  \item how to express type constraints
  \item how to chain contextual functions
  \item how to use IO
\end{itemize}
\end{minipage}
\end{center}
\end{frame}

\section{Typeclasses}

\begin{frame}[fragile]
\frametitle{Typeclasses}
\({c}%
  \<{.7\linewidth}%
  \begin{minipage}[t][.6\textheight]{\linewidth}%
\begin{onlyenv}<1>
\begin{code}
    class Show a where
        show :: a -> String
\end{code}
\end{onlyenv}
\begin{onlyenv}<2>
\begin{code}
    class Show a where
        show :: a -> String

    data Color = Red | Green | Blue
\end{code}
\end{onlyenv}
\begin{onlyenv}<3>
\begin{code}
    class Show a where
        show :: a -> String

    data Color = Red | Green | Blue

    instance Show Color where
        show Red   = "Red"
        show Green = "Green"
        show Blue  = "Blue"
\end{code}
\end{onlyenv}
  \end{minipage}%
  \>%
  \<{.3\linewidth}%
  \begin{minipage}[c][.8\textheight]{\linewidth}%
  \begin{center}%
    \vspace{2cm}
    \Image{4.2cm}{img/rainbow-5}
  \end{center}%
  \end{minipage}
  \>%
\)%
\end{frame}

\begin{frame}[fragile]
\frametitle{Constraints}
\({c}%
  \<{.26\linewidth}%
  \begin{minipage}[c][.8\textheight]{\linewidth}%
  \begin{center}%
    \vspace{2cm}
    \Image<1-3>{4cm}{img/rainbow-1}
    \Image<4-5>{4cm}{img/rainbow-defiant}
    \Image<6>{4cm}{img/rainbow-concerned}
    \Image<7>{4cm}{img/rainbow-ok}
  \end{center}%
  \end{minipage}
  \>%
  \<{.74\linewidth}%
  \begin{minipage}[t][.5\textheight]{\linewidth}%
\begin{onlyenv}<1>
\begin{code}
  show :: Show a => a -> String
\end{code}
\end{onlyenv}
\begin{onlyenv}<2>
\begin{code}
  show :: Show a => a -> String
  sum  :: Num a  => [a] -> a
\end{code}
\end{onlyenv}
\begin{onlyenv}<3>
\begin{code}
  show :: Show a => a -> String
  sum  :: Num a  => [a] -> a
  (==) :: Eq a   => a -> a -> Bool
\end{code}
\end{onlyenv}
\begin{onlyenv}<4>
\begin{code}
  show :: Show a => a -> String
  sum  :: Num a  => [a] -> a
  (==) :: Eq a   => a -> a -> Bool

  instance Show (Maybe a) where
\end{code}
\end{onlyenv}
\begin{onlyenv}<5>
\begin{code}
  show :: Show a => a -> String
  sum  :: Num a  => [a] -> a
  (==) :: Eq a   => a -> a -> Bool

  instance Show (Maybe a) where
      show Nothing  = "Nothing"
\end{code}
\end{onlyenv}
\begin{onlyenv}<6>
\begin{code}
  show :: Show a => a -> String
  sum  :: Num a  => [a] -> a
  (==) :: Eq a   => a -> a -> Bool

  instance Show (Maybe a) where
      show Nothing  = "Nothing"
      show (Just x) = "Just " ++ show x
\end{code}
\end{onlyenv}
\begin{onlyenv}<7>
\begin{code}
  show :: Show a => a -> String
  sum  :: Num a  => [a] -> a
  (==) :: Eq a   => a -> a -> Bool

  instance Show a => Show (Maybe a) where
      show Nothing  = "Nothing"
      show (Just x) = "Just " ++ show x
\end{code}
\end{onlyenv}
  \end{minipage}%
  \>%
\)%
\end{frame}

\begin{frame}[fragile]
\frametitle{Quick tour}
\frametitle<9->{Quick boring tour...}
\begin{center}
\begin{minipage}[t][.2\textheight]{.8\linewidth}
\begin{center}
\begin{tabular}{ c c c c c c }
\alt<2-5>{Show}{\color{lightgray}Show}&
\alt<4-5>{Read}{\color{lightgray}Read}&
\alt<6-8>{Eq}{\color{lightgray}Eq}&
\alt<9-10>{Ord}{\color{lightgray}Ord}&
\alt<11>{Bounded}{\color{lightgray}Bounded}&
\alt<12>{Enum}{\color{lightgray}Enum}
\end{tabular}
\hrule
\end{center}
\end{minipage}
~\\
\begin{minipage}[t][.6\textheight]{.8\linewidth}
\begin{onlyenv}<1>
\begin{code}
data Color = Red | Green | Blue
\end{code}
\end{onlyenv}
\begin{onlyenv}<2-5>
\begin{code}
data Color = Red | Green | Blue
\end{code}
\begin{flushleft}\footnotesize
~\\\uncover<2->{\texttt{> show Blue}}
~\\\uncover<3->{\texttt{"Blue"}}
~\\\uncover<4->{\texttt{> read "Green" :: Color}}
~\\\uncover<5->{\texttt{Green}}
\end{flushleft}
\end{onlyenv}
\begin{onlyenv}<6>
\begin{code}
data Color = Red | Green | Blue

class Eq a where
    (==) :: a -> a -> Bool
    (/=) :: a -> a -> Bool
\end{code}
\end{onlyenv}
\begin{onlyenv}<7>
\begin{code}
data Color = Red | Green | Blue

class Eq a where
    (==) :: a -> a -> Bool
    (/=) :: a -> a -> Bool
    a == b = not $ a /= b
    a /= b = not $ a == b
\end{code}
\end{onlyenv}
\begin{onlyenv}<8>
\begin{code}
data Color = Red | Green | Blue

instance Eq Color where
    Red   == Red   = True
    Green == Green = True
    Blue  == Blue  = True
    _     == _     = False
\end{code}
\end{onlyenv}
\begin{onlyenv}<9>
\begin{code}
data Color = Red | Green | Blue

class Eq a => Ord a where
    compare :: a -> a -> Ordering
    (<=)    :: a -> a -> Bool
    (>=)    :: a -> a -> Bool
    (<)     :: a -> a -> Bool
    (>)     :: a -> a -> Bool
    max     :: a -> a -> a
    min     :: a -> a -> a
\end{code}
\vspace{-1.9cm}
\hspace{6.2cm}
\Image{2.8cm}{img/fluttershy-bored-1}
\end{onlyenv}
\begin{onlyenv}<10>
\begin{code}
data Color = Red | Green | Blue

class Eq a => Ord a where
    compare :: a -> a -> Ordering
    (<=)    :: a -> a -> Bool
\end{code}
\vspace{0.6cm}
\hspace{6.2cm}
\Image{2.8cm}{img/fluttershy-bored-2}
\end{onlyenv}
\begin{onlyenv}<11>
\begin{code}
data Color = Red | Green | Blue

class Bounded a where
    minBound :: a
    maxBound :: a
\end{code}
\vspace{0.6cm}
\hspace{6.2cm}
\Image{2.8cm}{img/fluttershy-bored-3}
\end{onlyenv}
\begin{onlyenv}<12>
\begin{code}
data Color = Red | Green | Blue

class Enum a where
    succ           :: a -> a
    pred           :: a -> a
    toEnum         :: Int -> a
    fromEnum       :: a -> Int
    enumFrom       :: a -> [a]
    enumFromThen   :: a -> a -> [a]
    enumFromTo     :: a -> a -> [a]
    enumFromThenTo :: a -> a -> a -> [a]
\end{code}
\vspace{-2.4cm}
\hspace{6.2cm}
\Image{2.8cm}{img/fluttershy-bored-4}
\end{onlyenv}
\end{minipage}
\end{center}
\end{frame}

\begin{frame}[fragile]
\frametitle{Deriving}
\({c}%
  \<{.2\linewidth}%
  \begin{minipage}[t][.8\textheight]{\linewidth}%
  \begin{center}%
    \vspace{1cm}
    \Image<2>{1.8cm}{img/rainbow-3}
  \end{center}%
  \end{minipage}%
  \>%
  \<{.6\linewidth}%
  \begin{minipage}[t][.8\textheight]{\linewidth}%
  \begin{center}%
\vspace{2cm}
\begin{onlyenv}<1>
\begin{code}
data Color = Red | Green | Blue
\end{code}
\end{onlyenv}
\begin{onlyenv}<2>
\begin{code}
data Color = Red | Green | Blue
     deriving (Show,
               Read,
               Eq,
               Ord,
               Bounded,
               Enum)
\end{code}
\end{onlyenv}
  \end{center}%
  \end{minipage}%
  \>%
  \<{.2\linewidth}%
  \begin{minipage}[b][.8\textheight]{\linewidth}%
  \begin{center}%
    \Image<2>{3cm}{img/rainbow-2}
  \end{center}%
  \end{minipage}
  \>%
\)%
\end{frame}

\begin{frame}
\frametitle{We still don't know}
\begin{center}
\begin{minipage}[c]{.8\linewidth}
\begin{itemize}
  \item \st{how to extend our data types}
  \item \st{how to express type constraints}
  \item how to chain contextual functions
  \item how to use IO
\end{itemize}
\end{minipage}
\end{center}
\end{frame}


%% Links

\section{Monads}

\begin{frame}
\frametitle{Contexts / wrappers}
\begin{minipage}[c]{\linewidth}
  \begin{center}
    \begin{tabular}{ r c r }
      \uncover<1->{A          &~~~~~$\rightarrow$~~~~~& value} \\\\
      \uncover<2->{Maybe A    &~~~~~$\rightarrow$~~~~~& optional value} \\
      \uncover<3->{List A     &~~~~~$\rightarrow$~~~~~& repeated value} \\
      \uncover<4->{IO A       &~~~~~$\rightarrow$~~~~~& impure value}   \\\\
      \uncover<5->{C~~A       &~~~~~$\rightarrow$~~~~~& "contextual" value}   \\
    \end{tabular}
  \end{center}
\end{minipage}
\end{frame}

\begin{frame}
\frametitle{Similarities}
\begin{TopAlign}{.4}
\begin{itemize}
  \uncover<1->{\item Wrapping is trivial}
  \uncover<5->{\item Unwrapping is non-trivial / destructive / impossible}
  \uncover<9->{\item What we want: to apply functions \textbf{in the context}}
\end{itemize}
\end{TopAlign}
\({c}%
  \<{.6\linewidth}%
  \begin{minipage}[t][.5\textheight]{\linewidth}%
\begin{flushleft}
\begin{onlyenv}<2-4>
\uncover<2-4>{\footnotesize\texttt{~~~~\ic{wrap x = [x]}}\\}
\uncover<3-4>{\footnotesize\texttt{~~~~\ic{wrap x = Just x}}\\}
\uncover<4>{\footnotesize\texttt{~~~~\ic{wrap x = return x}}\\}
\end{onlyenv}
\begin{onlyenv}<6-8>
\uncover<6->{\footnotesize\texttt{~~~~\ic{unwrap Nothing = ???}}\\}
\uncover<7->{\footnotesize\texttt{~~~~\ic{unwrap [1,2,3] = ???}}\\}
\uncover<8->{\footnotesize\texttt{~~~~\ic{unwrap getLine =} {\color{red}NOPE}}}\\
\end{onlyenv}
\end{flushleft}
  \end{minipage}%
  \>%
  \<{.4\linewidth}%
  \begin{minipage}[t][.5\textheight]{\linewidth}%
  \begin{flushleft}%
    \vspace{-0.75cm}
    \Image<1>{4cm}{img/applejack-wink}
    \Image<2-3>{4cm}{img/applejack-ok}
    \Image<4>{4cm}{img/applejack-scared}
    \Image<5>{4cm}{img/applejack-intrigued}
    \Image<6-7>{4cm}{img/applejack-frown}
    \Image<8>{4cm}{img/applejack-frightened}
    \Image<9>{4cm}{img/applejack-intrigued}
  \end{flushleft}%
  \end{minipage}
  \>%
\)%
\end{frame}

\begin{frame}
\frametitle{Context functions}
\begin{minipage}[c][.3\textheight]{\linewidth}%
\begin{center}
  Three standard functions to deal with contexts
\end{center}
\end{minipage}
\({c}%
  \<{.61\linewidth}%
  \begin{minipage}[t][.7\textheight]{\linewidth}%
\begin{flushleft}
\vspace{1cm}
\upc<2->{~\\~~fmap :: ~~\ic{(a -> b) -> c a -> c b}\\}
\upc<3->{~\\~~ap~~ :: \ic{c (a -> b) -> c a -> c b}\\}
\upc<4->{~\\~~bind :: \ic{(a -> c b) -> c a -> c b}\\}
\end{flushleft}
  \end{minipage}%
  \>%
  \<{.39\linewidth}%
  \begin{minipage}[t][.7\textheight]{\linewidth}%
  \begin{flushleft}%
    \Image<2->{3cm}{img/scootaloo-dwi}~\\
    \vspace{-2.8cm}
    \hspace{1cm}
    \Image<3->{3.4cm}{img/applebloom-dwi}~\\
    \vspace{-2.4cm}
    \hspace{2cm}
    \Image<4->{3cm}{img/sweetiebelle-dwi}~\\
  \end{flushleft}%
  \end{minipage}
  \>%
\)%
\end{frame}

\begin{frame}
\frametitle{fmap}
\begin{TopAlign}{.4}
\begin{itemize}
  \uncover<1->{\item fmap is like map\ldots}
  \uncover<3->{\item but generalised to all contexts}
  \uncover<13->{\item and it also has an operator version}
\end{itemize}
\end{TopAlign}
\begin{center}%
\begin{minipage}[t][.6\textheight]{.72\linewidth}%
\begin{center}%
\opc<2>{~\ic{map :: (a -> b) -> [a] -> [b]}}%
\opc<3>{\ic{fmap :: (a -> b) -> c a -> c b}}%
\begin{onlyenv}<4-13>
\begin{minipage}[t]{\linewidth}%
\begin{flushleft}
  \upc<4-> {\ic{fmap show [1, 2, 3] =} \uic<5->{["1", "2", "3"]}\\}
  \upc<6-> {\ic{fmap show (Just 42) =} \uic<7->{Just "42"}\\}
  \upc<8-> {\ic{fmap show Nothing}\uic<9->{   = Nothing}\\}
  \upc<10->{\ic{fmap length getLine =} \upc<11->{\shruggie} \uic<12->{-- IO Int}\\}
\end{flushleft}
\end{minipage}
\end{onlyenv}
\begin{onlyenv}<14>
\begin{minipage}[t]{\linewidth}%
\begin{flushleft}
  \pc{\ic{show  <$> [1, 2, 3] =} \ic{["1", "2", "3"]}\\}
  \pc{\ic{show  <$> (Just 42) =} \ic{Just "42"}\\}
  \pc{\ic{show  <$> Nothing}\ic{   = Nothing}\\}
  \pc{\ic{length <$>  getLine =} \pc{\shruggie} \ic{-- IO Int}\\}
\end{flushleft}
\end{minipage}
\end{onlyenv}
\end{center}
\end{minipage}%
\end{center}
\end{frame}

\begin{frame}
\frametitle{ap(ply)}
\begin{TopAlign}{.4}
\begin{itemize}
  \uncover<1->{\item what about functions with several arguments?}
  \uncover<5->{\item apply to the rescue!}
  \uncover<8->{\item also defined as an operator}
\end{itemize}
\end{TopAlign}
\begin{center}%
\begin{minipage}[t][.6\textheight]{.72\linewidth}%
\begin{onlyenv}<1->
\begin{flushleft}
  \upc<5->  {\ic{ap :: c (a -> b) -> c a -> c b}\\}
  \opc<2-9> {\\\ic{fmap (+) (Just 3) =} \uic<3->{Just (3+)}\\}
  \opc<10>  {\\\ic{(+) <$> Just 3 =} \uic<3->{Just (3+)}\\}
  \opc<4-5> {\ic{Just (3+) :: Maybe (Int -> Int)}\\}
  \opc<6-8> {\ic{ap (Just (3+)) (Just 39) =} \uic<7->{Just 42}\\}
  \opc<9-10>{\ic{Just (3+) <*> Just 39 = Just 42}\\}
  \opc<11-> {\\\ic{(+)} \uic<12->{<$> Just 3} \uic<13->{<*> Just 39} \uic<14->{= Just 42}\\}%$
\end{flushleft}
\end{onlyenv}
\end{minipage}%
\end{center}
\end{frame}

\begin{frame}[fragile]
\frametitle{bind}
\begin{TopAlign}{.32}
\begin{itemize}
  \uncover<1->{\item what about chaining functions?}
  \uncover<10->{\item bind to the rescue!}
  \uncover<13->{\item of course, also has an operator}
\end{itemize}
\end{TopAlign}
\begin{center}%
\begin{minipage}[t][.6\textheight]{\linewidth}%
\begin{onlyenv}<2>
\begin{code}
          div2 :: Int -> Maybe Int
\end{code}
\end{onlyenv}
\begin{onlyenv}<3>
\begin{code}
          div2 :: Int -> Maybe Int

          div2 42 = Just 21
\end{code}
\end{onlyenv}
\begin{onlyenv}<4>
\begin{code}
          div2 :: Int -> Maybe Int

          div2 42 = Just 21
          div2 21 = Nothing
\end{code}
\end{onlyenv}
\begin{onlyenv}<5>
\begin{code}
          div2 :: Int -> Maybe Int


          div4 :: Int -> Maybe Int
\end{code}
\end{onlyenv}
\begin{onlyenv}<6>
\begin{code}
          div2 :: Int -> Maybe Int


          div4 :: Int -> Maybe Int
          div4 x =
\end{code}
\end{onlyenv}
\begin{onlyenv}<7>
\begin{code}
          div2 :: Int -> Maybe Int


          div4 :: Int -> Maybe Int
          div4 x = let y = div2 x
                   in
\end{code}
\end{onlyenv}
\begin{onlyenv}<8>
\begin{code}
          div2 :: Int -> Maybe Int


          div4 :: Int -> Maybe Int
          div4 x = let y = div2 x
                   in fmap div2 y
\end{code}
\end{onlyenv}
\begin{onlyenv}<9>
\begin{code}
          div2 :: Int -> Maybe Int


          div4 :: Int -> Maybe Int
          div4 x = let y = div2 x -- Maybe Int
                   in fmap div2 y -- Maybe (Maybe Int)
\end{code}
\end{onlyenv}
\begin{onlyenv}<10>
\begin{code}
          div2 :: Int -> Maybe Int
          bind :: (a -> c b) -> c a -> c b

          div4 :: Int -> Maybe Int
          div4 x = let y = div2 x -- Maybe Int
                   in fmap div2 y -- Maybe (Maybe Int)
\end{code}
\end{onlyenv}
\begin{onlyenv}<11>
\begin{code}
          div2 :: Int -> Maybe Int
          bind :: (a -> c b) -> c a -> c b

          div4 :: Int -> Maybe Int
          div4 x = let y = div2 x -- Maybe Int
                   in bind div2 y -- Maybe Int
\end{code}
\end{onlyenv}
\begin{onlyenv}<12>
\begin{code}
          div2 :: Int -> Maybe Int
          bind :: (a -> c b) -> c a -> c b

          div4 :: Int -> Maybe Int
          div4 x = bind div2 $ div2 x
\end{code}%$
\end{onlyenv}
\begin{onlyenv}<13>
\begin{code}
          div2 :: Int -> Maybe Int
          (>>=) :: c a -> (a -> c b) -> c b

          div4 :: Int -> Maybe Int
          div4 x = bind div2 $ div2 x
\end{code}%$
\end{onlyenv}
\begin{onlyenv}<14>
\begin{code}
          div2 :: Int -> Maybe Int
          (>>=) :: c a -> (a -> c b) -> c b

          div4 :: Int -> Maybe Int
          div4 x = div2 x >>= div2
\end{code}
\end{onlyenv}
\begin{onlyenv}<15>
\begin{code}
          div2 :: Int -> Maybe Int
          (>>=) :: c a -> (a -> c b) -> c b

          div8 :: Int -> Maybe Int
          div8 x = div2 x >>= div2 >>= div2
\end{code}
\end{onlyenv}
\begin{onlyenv}<16>
\begin{code}
          div2 :: Int -> Maybe Int
          (>>=) :: c a -> (a -> c b) -> c b

          div16 :: Int -> Maybe Int
          div16 x = div2 x >>= div2 >>= div2 >>= div2
\end{code}
\end{onlyenv}
\end{minipage}%
\end{center}
\end{frame}

\begin{frame}[fragile]
\frametitle{bind - continued}
\begin{minipage}[t][.8\textheight]{\linewidth}
\vspace{-0.8cm}
\begin{onlyenv}<1>
\begin{code}
getUser        :: Id   -> Maybe User
getNextOfKin   :: User -> Maybe Id
getPhoneNumber :: User -> Maybe PhoneNumber
\end{code}
\end{onlyenv}
\begin{onlyenv}<2>
\begin{code}
getUser        :: Id   -> Maybe User
getNextOfKin   :: User -> Maybe Id
getPhoneNumber :: User -> Maybe PhoneNumber


getNextOfKinPhoneNumber :: Id -> Maybe PhoneNumber
\end{code}
\end{onlyenv}
\begin{onlyenv}<3>
\begin{code}
getUser        :: Id   -> Maybe User
getNextOfKin   :: User -> Maybe Id
getPhoneNumber :: User -> Maybe PhoneNumber
bind           :: (a -> c b) -> c a -> c b

getNextOfKinPhoneNumber :: Id -> Maybe PhoneNumber
\end{code}
\end{onlyenv}
\begin{onlyenv}<4>
\begin{code}
getUser        :: Id   -> Maybe User
getNextOfKin   :: User -> Maybe Id
getPhoneNumber :: User -> Maybe PhoneNumber
(>>=)          :: c a -> (a -> c b) -> c b

getNextOfKinPhoneNumber :: Id -> Maybe PhoneNumber
\end{code}
\end{onlyenv}
\begin{onlyenv}<5>
\begin{code}
getUser        :: Id   -> Maybe User
getNextOfKin   :: User -> Maybe Id
getPhoneNumber :: User -> Maybe PhoneNumber
(>>=)          :: c a -> (a -> c b) -> c b

getNextOfKinPhoneNumber :: Id -> Maybe PhoneNumber
getNextOfKinPhoneNumber uid =
\end{code}
\end{onlyenv}
\begin{onlyenv}<6>
\begin{code}
getUser        :: Id   -> Maybe User
getNextOfKin   :: User -> Maybe Id
getPhoneNumber :: User -> Maybe PhoneNumber
(>>=)          :: c a -> (a -> c b) -> c b

getNextOfKinPhoneNumber :: Id -> Maybe PhoneNumber
getNextOfKinPhoneNumber uid =
      getUser uid    -- Maybe User
\end{code}
\end{onlyenv}
\begin{onlyenv}<7>
\begin{code}
getUser        :: Id   -> Maybe User
getNextOfKin   :: User -> Maybe Id
getPhoneNumber :: User -> Maybe PhoneNumber
(>>=)          :: c a -> (a -> c b) -> c b

getNextOfKinPhoneNumber :: Id -> Maybe PhoneNumber
getNextOfKinPhoneNumber uid =
      getUser uid    -- Maybe User
  >>= getNextOfKin   -- Maybe Id
\end{code}
\end{onlyenv}
\begin{onlyenv}<8>
\begin{code}
getUser        :: Id   -> Maybe User
getNextOfKin   :: User -> Maybe Id
getPhoneNumber :: User -> Maybe PhoneNumber
(>>=)          :: c a -> (a -> c b) -> c b

getNextOfKinPhoneNumber :: Id -> Maybe PhoneNumber
getNextOfKinPhoneNumber uid =
      getUser uid    -- Maybe User
  >>= getNextOfKin   -- Maybe Id
  >>= getUser        -- Maybe User
\end{code}
\end{onlyenv}
\begin{onlyenv}<9>
\begin{code}
getUser        :: Id   -> Maybe User
getNextOfKin   :: User -> Maybe Id
getPhoneNumber :: User -> Maybe PhoneNumber
(>>=)          :: c a -> (a -> c b) -> c b

getNextOfKinPhoneNumber :: Id -> Maybe PhoneNumber
getNextOfKinPhoneNumber uid =
      getUser uid    -- Maybe User
  >>= getNextOfKin   -- Maybe Id
  >>= getUser        -- Maybe User
  >>= getPhoneNumber
\end{code}
\end{onlyenv}
\end{minipage}
\begin{center}
\begin{onlyenv}<1>
\vspace{-3cm}
\hspace{4.5cm}
\Image{4cm}{img/fluttershy-walking}%
\end{onlyenv}
\begin{onlyenv}<2>
\vspace{-3.4cm}
\hspace{5.7cm}
\Image{4cm}{img/fluttershy-scared2}%
\end{onlyenv}
\begin{onlyenv}<3-6>
\vspace{-1.5cm}
\hspace{5.7cm}
\Image{2.2cm}{img/fluttershy-flat}%
\end{onlyenv}
\begin{onlyenv}<7>
\vspace{-2.2cm}
\hspace{5.7cm}
\Image{3cm}{img/fluttershy-scared}%
\end{onlyenv}
\begin{onlyenv}<8>
\vspace{-3.2cm}
\hspace{5.7cm}
\Image{3.9cm}{img/fluttershy-curious}%
\end{onlyenv}
\begin{onlyenv}<9>
\vspace{-2.88cm}
\hspace{6.1cm}
\Image{3.5cm}{img/fluttershy-ok}%
\end{onlyenv}
\end{center}
\end{frame}

\def\Sep{~~~~~&~~~~~}
\begin{frame}
\frametitle{The typeclasses}
\begin{onlyenv}<1-2>
  \begin{center}
    \pc{fmap :: ~~\ic{(a -> b) -> c a -> c b}\\}
    \pc{ap~~ :: \ic{c (a -> b) -> c a -> c b}\\}
    \pc{bind :: \ic{(a -> c b) -> c a -> c b}\\}

    \uncover<2>{~\\But what about the typeclasses?}
  \end{center}
\end{onlyenv}
\begin{onlyenv}<3-7,9->
  \begin{center}
    ~\\
    \begin{tabular}{ c l c l c }
\uncover<3->{                  \Sep\pc{fmap}\Sep$\rightarrow$\Sep\upc<4->{Functor    }\Sep                  \\}
\uncover<5->{$^\downrcurvearrow$\Sep\pc{ap}  \Sep$\rightarrow$\Sep\upc<6->{Applicative}\Sep$^\downlcurvearrow$\\}
\uncover<7->{$^\downrcurvearrow$\Sep\pc{bind}\Sep$\rightarrow$\Sep\upc<8->{Monad}      \Sep$^\downlcurvearrow$\\}
    \end{tabular}
  \end{center}
\end{onlyenv}
\begin{onlyenv}<8>
\begin{minipage}[c][.8\textheight]{\linewidth}
\begin{center}
{\fontsize{150}{60}\selectfont MONAD}\\
\end{center}
\end{minipage}
\begin{center}
\vspace{-.5cm}
\hspace{-2cm}
\Image{2cm}{img/pinkie-dwi}
\end{center}
\end{onlyenv}
\end{frame}

\begin{frame}
\frametitle{We still don't know}
\begin{center}
\begin{minipage}[c]{.8\linewidth}
\begin{itemize}
  \item \st{how to extend our data types}
  \item \st{how to express type constraints}
  \item \st{how to chain contextual functions}
  \item how to use IO
\end{itemize}
\end{minipage}
\end{center}
\end{frame}


%% Links

\section{Monadic syntax}

\begin{frame}[fragile]
\frametitle{Monad}
\begin{minipage}[t][.5\textheight]{\linewidth}
\begin{center}
  \begin{onlyenv}<1>
  \begin{code}
    class Applicative m => Monad m where
        return :: a -> m a
        (>>=)  :: m a -> (a -> m b) -> m b
  \end{code}
  \end{onlyenv}
  \begin{onlyenv}<2>
  \begin{code}
    class Applicative m => Monad m where
        return :: a -> m a
        (>>=)  :: m a -> (a -> m b) -> m b

    (>>)     :: m a -> m b -> m b
    (>=>)    :: (a -> m b) -> (b -> m c) -> (a -> m c)
  \end{code}
  \end{onlyenv}
  \begin{onlyenv}<3>
  \begin{code}
    class Applicative m => Monad m where
        return :: a -> m a
        (>>=)  :: m a -> (a -> m b) -> m b

    (>>)     :: m a -> m b -> m b
    (>=>)    :: (a -> m b) -> (b -> m c) -> (a -> m c)
    sequence :: (Traversable t) => t (m a) -> m (t a)
  \end{code}
  \end{onlyenv}
\end{center}
\end{minipage}
\begin{minipage}[t][.3\textheight]{\linewidth}
\begin{center}
  \only<1>{The basic building block\\}
  \only<2>{Powerful abstractions!\\}
  \only<3>{\ldots{}that could use some sugar\\}
\begin{onlyenv}<1>
\hspace{5cm}
\Image{3.2cm}{img/twilight-crazy}%
\end{onlyenv}
\begin{onlyenv}<2>
\hspace{5cm}
\Image{3.2cm}{img/twilight-wut}%
\end{onlyenv}
\begin{onlyenv}<3>
\vspace{0.6cm}
\hspace{5cm}
\Image{2.6cm}{img/twilight-coffee}%
\end{onlyenv}
\end{center}
\end{minipage}
\end{frame}

\begin{frame}[fragile]
\frametitle{Do notation}
\begin{center}
\begin{minipage}[t][\textheight]{.8\linewidth}
\vspace{-0.8cm}
\begin{onlyenv}<1>
\begin{code}
getFirstName :: IO String
getLastName  :: IO String
greetUser    :: String -> String -> IO ()
main         :: IO ()

main =
\end{code}
\end{onlyenv}
\begin{onlyenv}<2>
\begin{code}
getFirstName :: IO String
getLastName  :: IO String
greetUser    :: String -> String -> IO ()
main         :: IO ()

main =
    getFirstName
\end{code}
\end{onlyenv}
\begin{onlyenv}<3>
\begin{code}
getFirstName :: IO String
getLastName  :: IO String
greetUser    :: String -> String -> IO ()
main         :: IO ()

main =
    getFirstName >>=
\end{code}
\end{onlyenv}
\begin{onlyenv}<4>
\begin{code}5
getFirstName :: IO String
getLastName  :: IO String
greetUser    :: String -> String -> IO ()
main         :: IO ()

main =
    getFirstName >>= \ firstname ->
\end{code}
\end{onlyenv}
\begin{onlyenv}<5>
\begin{code}
getFirstName :: IO String
getLastName  :: IO String
greetUser    :: String -> String -> IO ()
main         :: IO ()

main =
    getFirstName >>= \ firstname ->
        getLastName
\end{code}
\end{onlyenv}
\begin{onlyenv}<6>
\begin{code}
getFirstName :: IO String
getLastName  :: IO String
greetUser    :: String -> String -> IO ()
main         :: IO ()

main =
    getFirstName >>= \ firstname ->
        getLastName >>=
\end{code}
\end{onlyenv}
\begin{onlyenv}<7>
\begin{code}
getFirstName :: IO String
getLastName  :: IO String
greetUser    :: String -> String -> IO ()
main         :: IO ()

main =
    getFirstName >>= \ firstname ->
        getLastName >>= \ lastname ->
\end{code}
\end{onlyenv}
\begin{onlyenv}<8>
\begin{code}
getFirstName :: IO String
getLastName  :: IO String
greetUser    :: String -> String -> IO ()
main         :: IO ()

main =
    getFirstName >>= \ firstname ->
        getLastName >>= \ lastname ->
            greetUser firstname lastname
\end{code}
\end{onlyenv}
\begin{onlyenv}<9>
\begin{code}
getFirstName :: IO String
getLastName  :: IO String
greetUser    :: String -> String -> IO ()
main         :: IO ()

main =
    getFirstName       firstname
        getLastName       lastname
            greetUser firstname lastname
\end{code}
\end{onlyenv}
\begin{onlyenv}<10>
\begin{code}
getFirstName :: IO String
getLastName  :: IO String
greetUser    :: String -> String -> IO ()
main         :: IO ()

main =
    getFirstName ->    firstname
        getLastName ->    lastname
            greetUser firstname lastname
\end{code}
\end{onlyenv}
\begin{onlyenv}<11>
\begin{code}
getFirstName :: IO String
getLastName  :: IO String
greetUser    :: String -> String -> IO ()
main         :: IO ()

main =
    getFirstName >>= \ firstname ->
        getLastName >>= \ lastname ->
            greetUser firstname lastname

main =
\end{code}
\end{onlyenv}
\begin{onlyenv}<12>
\begin{code}
getFirstName :: IO String
getLastName  :: IO String
greetUser    :: String -> String -> IO ()
main         :: IO ()

main =
    getFirstName >>= \ firstname ->
        getLastName >>= \ lastname ->
            greetUser firstname lastname

main = do
\end{code}
\end{onlyenv}
\begin{onlyenv}<13>
\begin{code}
getFirstName :: IO String
getLastName  :: IO String
greetUser    :: String -> String -> IO ()
main         :: IO ()

main =
    getFirstName >>= \ firstname ->
        getLastName >>= \ lastname ->
            greetUser firstname lastname

main = do
    firstname <- getFirstName
\end{code}
\end{onlyenv}
\begin{onlyenv}<14>
\begin{code}
getFirstName :: IO String
getLastName  :: IO String
greetUser    :: String -> String -> IO ()
main         :: IO ()

main =
    getFirstName >>= \ firstname ->
        getLastName >>= \ lastname ->
            greetUser firstname lastname

main = do
    firstname <- getFirstName
    lastname  <- getLastName
\end{code}
\end{onlyenv}
\begin{onlyenv}<15>
\begin{code}
getFirstName :: IO String
getLastName  :: IO String
greetUser    :: String -> String -> IO ()
main         :: IO ()

main =
    getFirstName >>= \ firstname ->
        getLastName >>= \ lastname ->
            greetUser firstname lastname

main = do
    firstname <- getFirstName
    lastname  <- getLastName
    greetUser firstname lastname
\end{code}
\end{onlyenv}
\end{minipage}
\end{center}
\end{frame}

\begin{frame}
\frametitle{Recap}
\begin{itemize}
\item typeclasses to extend our data types
\item typeclasses to express type constraints
\item monadic operators to chain contextual functions
\item do notation to use IO
\end{itemize}
\end{frame}



%% Links

\section{The end}

\begin{frame}
  \begin{center}
    \vspace{0.20cm}
    \includegraphics[width=0.9\textwidth]{img/pinkie-festive}
  \end{center}
\end{frame}

\begin{frame}
  \frametitle{Links}
  \begin{itemize}
  \item \href{http://adit.io/posts/2013-04-17-functors,_applicatives,_and_monads_in_pictures.html}{adit.io} (functors, applicatives, and monads in pictures)
  \item \href{http://dev.stephendiehl.com/hask/}{dev.stephendiehl.com/hask/} (what I wish I knew)
  \end{itemize}
\end{frame}



%%%%%%%%%%%%%%%%%%%%%%%%%%%%%%%%%%%%%%%%%%%%%%%%%%%%%%%%%%%%%%%%%%%%%%%%%%%%
%% End

\end{document}
